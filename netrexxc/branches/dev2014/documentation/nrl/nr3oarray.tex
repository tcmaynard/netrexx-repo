\subsection{Arrays}\label{refoarray}
\index{Overview,Arrays}
\index{Arrays,overview}
:p.
NetRexx also supports fixed-size \emph{arrays}.  These are an
ordered set of items, indexed by integers.  To use an array, you first
have to construct it; an individual item may then be selected by an
index whose value must be in the range \textbf{0}
through \textbf{n-1}, where \textbf{n} is the number of items in the
array:
\index{Example,arrays}
\begin{verbatim}
array=String[3]        -- make an array of three Strings
array[0]='String one'  -- set each array item
array[1]='Another string'
array[2]='foobar'
loop i=0 to 2          -- display the items
  say array[i]
  end
\end{verbatim}
:p.
This example also shows NetRexx \emph{line comments}; the sequence
"\textbf{--}" (outside of literal strings or
"\textbf{/*}" comments) indicates that the remainder of the
line is not part of the program and is commentary.
:p.
NetRexx makes it easy to initialize arrays: a term which is a list of
one or more expressions, enclosed in brackets, defines an array.  Each
expression initializes an element of the array.
For example:
\begin{verbatim}
words=['Ogof', 'Ffynnon', 'Ddu']
\end{verbatim}
:pc.would set :m.words:em. to refer to an array of three elements, each
referring to a string.  So, for example, the instruction:
\begin{verbatim}
say words[1]
\end{verbatim}
:pc.would then display :m.Ffynnon:em..
