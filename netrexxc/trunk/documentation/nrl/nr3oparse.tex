\subsection{Parsing strings}\label{refoparse}
\index{Overview,parsing}
\index{Parsing,overview}
:p.
The previous section described some of the string-handling facilities
available; NetRexx also provides string parsing, which is an easy way of
breaking up strings of characters using simple pattern matching.
:p.
A \texttt{parse} instruction first specifies the string to be parsed.
This can be any term, but is often taken simply from a variable.  The
term is followed by a \emph{template} which describes how the string
is to be split up, and where the pieces are to be put.
:h4.Parsing into words
:p.
The simplest form of parsing template consists of a list of variable
names.  The string being parsed is split up into words (sequences of
characters separated by blanks), and each word from the string is
assigned (copied) to the next variable in turn, from left to right.  The
final variable is treated specially in that it will be assigned a copy
of whatever is left of the original string and may therefore contain
several words. For example, in:
\begin{verbatim}
parse 'This is a sentence.'   v1 v2 v3
\end{verbatim}
:p.
the variable \textbf{v1} would be assigned the value
"\textbf{This}", \textbf{v2} would be assigned the value
"\textbf{is}", and \textbf{v3} would be assigned the value
"\textbf{a sentence.}".
:h4.Literal patterns
:p.
A literal string may be used in a template as a pattern to split up the
string.  For example
\begin{verbatim}
parse 'To be, or not to be?'   w1 ',' w2 w3 w4
\end{verbatim}
:p.
would cause the string to be scanned for the comma, and then split at
that point; each section is then treated in just the same way as the
whole string was in the previous example.
:p.
Thus, \textbf{w1} would be set to
"\textbf{To be}", \textbf{w2} and \textbf{w3} would be
assigned the values "\textbf{or}" and "\textbf{not}",
and \textbf{w4} would be assigned the remainder: "\textbf{to
be?}".  Note that the pattern itself is not assigned to any
variable.
:p.
The pattern may be specified as a variable, by putting the variable name
in parentheses.  The following instructions:
\begin{verbatim}
comma=','
parse 'To be, or not to be?'   w1 (comma) w2 w3 w4
\end{verbatim}
:p.
therefore have the same effect as the previous example.
:h4.Positional patterns
:p.
The third kind of parsing mechanism is the numeric positional pattern.
This allows strings to be parsed using column positions.
