\subsection{Tracing}\label{refotrace}
\index{Overview,tracing}
\index{Tracing,overview}
:p.
NetRexx tracing is defined as part of the language.
The flow of execution of programs may be traced, and this trace can be
viewed as it occurs (or captured in a file).  The trace can show each
clause as it is executed, and optionally show the results of
expressions, \&  For example, the \texttt{trace results} in the
program ":m.trace1.nrx:em.":
\index{Example,trace}
\index{Example,program}
\begin{verbatim}
trace results
number=1/7
parse number before '.' after
say after'.'before
\end{verbatim}
:p.
would result in:
\begin{verbatim}
   --- trace1.nrx
 2 *=* number=1/7
   >v> number "0.142857143"
 3 *=* parse number before '.' after
   >v> before "0"
   >v> after "142857143"
 4 *=* say after'.'before
   >>> "142857143.0"
142857143.0
\end{verbatim}
:pc.
where the line marked with ":m.---:em." indicates the context
of the trace, lines marked with ":m.*=*:em." are the
instructions in the program, lines with ":m.>v>:em."
show results assigned to local variables, and lines with
":m.>>>:em." show results of un-named expressions.
:p.
Further, \texttt{trace methods} lets you trace the use of all methods in
a class, along with the values of the arguments passed to each method.
Here's the result of adding :m.trace methods:em. to the Oblong class
shown earlier and then running :m.tryOblong:em.:
\begin{verbatim}
    --- Oblong.nrx
  8 *=*     method Oblong(newwidth, newheight)
    >a> newwidth "5"
    >a> newheight "3"
 26 *=*     method print
Oblong 5 x 3
 20 *=*     method relsize(relwidth, relheight)-
 21 *-*                   returns Oblong
    >a> relwidth "1"
    >a> relheight "1"
 26 *=*     method print
Oblong 6 x 4
 10 *=*     method Oblong(newwidth, newheight)
    >a> newwidth "1"
    >a> newheight "2"
 26 *=*     method print
Oblong 1 x 2
\end{verbatim}
:pc.where
lines with ":m.>a>:em." show the names and values of the
arguments.
:p.It's often useful to be able to find out when (and where) a
variable's value is changed.  The \texttt{trace var} instruction does
just that; it adds names to or removes names from a list of monitored
variables. If the name of a variable in the current class or method is
in the list, then \texttt{trace results} is turned on for any
assignment, \texttt{loop}, or \texttt{parse} instruction that assigns a
new value to the named variable.
:p.
Variable names to be added to the list are specified by listing them
after the \texttt{var} keyword.
Any name may be optionally prefixed by a :m.-:em. sign., which indicates
that the variable is to be removed from the list.
:p.For example, the program ":m.trace2.nrx:em.":
\index{Example,trace}
\index{Example,program}
\begin{verbatim}
trace var a b
-- now variables a and b will be traced
a=3
b=4
c=5
trace var -b c
-- now variables a and c will be traced
a=a+1
b=b+1
c=c+1
say a b c
\end{verbatim}
:pc.
would result in:
\begin{verbatim}
    --- trace2.nrx
  3 *=* a=3
    >v> a "3"
  4 *=* b=4
    >v> b "4"
  8 *=* a=a+1
    >v> a "4"
 10 *=* c=c+1
    >v> c "6"
4 5 6
\end{verbatim}
