\section{The R\textsc{exx}Set class}\label{refnlrset}
\index{netrexx.lang,R\textsc{exx}Set class}
 
The \textbf{R\textsc{exx}Set} class is used to provide the numeric settings for
the methods described in the section R\textsc{exx} (see page \pageref{refrexxops}) 
arithmetic methods.
When provided, a R\textsc{exx}Set Object supplies the \keyword{numeric} settings
for the operation; when \textbf{null} is provided then the default
settings are used (\keyword{digits}=\textbf{9},
\keyword{form}=\keyword{SCIENTIFIC}).
\subsection{Public properties}\label{publicproperties}
\index{netrexx.lang,R\textsc{exx}Set properties}
 
These properties supply the numeric settings and certain values they may
take.  After construction, the \keyword{digits} and \keyword{form} values
should only be changed by using the \keyword{setDigits} and
\keyword{setForm} methods.
\begin{description}
\item{DEFAULT\_DIGITS}
\index{DEFAULT\_DIGITS property,}
\index{Property,DEFAULT\_DIGITS}

A constant of type \textbf{int} that describes the default number of
digits for a numeric operation (9).
\item{DEFAULT\_FORM}
\index{DEFAULT\_FORM property,}
\index{Property,DEFAULT\_FORM}

A constant of type \textbf{byte} that describes the default exponential
format for a numeric operation (\keyword{SCIENTIFIC}).
\item{digits}
\index{digits property,}
\index{Property,digits}

A value of type \textbf{int} that describes the numeric digits to be
used for a numeric operation.  The  R\textsc{exx} arithmetic (see page \pageref{refrexxops}) 
methods:ea. use this value to determine the significance of results.
\keyword{digits} must always be greater than zero.
\item{ENGINEERING}
\index{ENGINEERING property,}
\index{Property,ENGINEERING}

A constant of type \textbf{byte} that signifies that engineering
exponential formatting should be used for a numeric operation.
\item{form}
\index{form property,}
\index{Property,form}

A value of type \textbf{byte} that describes the exponential format to
be used for a numeric operation.  The  R\textsc{exx} arithmetic (see page \pageref{refrexxops}) 
methods:ea. use this value to determine the formatting of results that
require an exponent.
\keyword{form} must be either \keyword{ENGINEERING} or \keyword{SCIENTIFIC}.
\item{SCIENTIFIC}
\index{SCIENTIFIC property,}
\index{Property,SCIENTIFIC}

A constant of type \textbf{byte} that signifies that scientific
exponential formatting should be used for a numeric operation.
\end{description}
\subsection{Constructors}\label{constructors}
\index{netrexx.lang,R\textsc{exx}Set constructors}
 
These constructors are used to set the initial values of a R\textsc{exx}Set
object.
\begin{description}
\item{R\textsc{exx}Set()}
\index{R\textsc{exx}Set() constructor,}
\index{Constructor,R\textsc{exx}Set()}

Constructs a R\textsc{exx}Set object which has default \keyword{digits} and
\keyword{form} properties.
\item{R\textsc{exx}Set(newdigits=int)}
\index{R\textsc{exx}Set(int) constructor,}
\index{Constructor,R\textsc{exx}Set(int)}

Constructs a R\textsc{exx}Set object which has its \keyword{digits} property set
to \emph{newdigits} and its \keyword{form} property set
to \keyword{DEFAULT\_DIGITS}.
\item{R\textsc{exx}Set(newdigits=int, newform=byte)}
\index{R\textsc{exx}Set(int,byte) constructor,}
\index{Constructor,R\textsc{exx}Set(int,byte)}

Constructs a R\textsc{exx}Set object which has its \keyword{digits} property set
to \emph{newdigits} and its \keyword{form} property set to
\emph{newform}.
\item{R\textsc{exx}Set(arg=R\textsc{exx}Set)}
\index{R\textsc{exx}Set(R\textsc{exx}Set) constructor,}
\index{Constructor,R\textsc{exx}Set(R\textsc{exx}Set)}

Constructs a R\textsc{exx}Set object which is copy of \emph{arg}, which is of
type \textbf{netrexx.lang.R\textsc{exx}Set}.
\emph{arg} must not be \textbf{null}.
\end{description}

\index{netrexx.lang,R\textsc{exx}Set methods}

\subsection{Methods}\label{rexxsetmethods}
 
The R\textsc{exx}Set class has the following additional methods:
\begin{description}
\item{formword()}
\index{formword() method,}
\index{Method,formword()}

Returns a string of type \textbf{netrexx.lang.R\textsc{exx}} that describes the
\keyword{form} property.  This will either be the string \textbf{'engineering'}
or the string \textbf{'scientific'}, corresponding to the \keyword{form}
value \keyword{ENGINEERING} or \keyword{SCIENTIFIC}, respectively.
\item{setDigits(newdigits=R\textsc{exx})}
\index{setDigits(R\textsc{exx}) method,}
\index{Method,setDigits(R\textsc{exx})}

Sets the \keyword{digits} value for the \textbf{R\textsc{exx}Set} object, from
\emph{newdigits}, after rounding and checking as defined for the
\keyword{numeric} instruction; \emph{newdigits} must be a positive
whole number with no more than nine digits.
No value is returned.
\item{setForm(newformword=R\textsc{exx})}
\index{setForm(R\textsc{exx}) method,}
\index{Method,setForm(R\textsc{exx})}

Sets the \keyword{form} value for the \textbf{R\textsc{exx}Set} object, from
\emph{newformword}.
This must equal either the string \textbf{'engineering'} or the
string \textbf{'scientific'}, corresponding to the \keyword{form}
value \keyword{ENGINEERING} or \keyword{SCIENTIFIC}, respectively.
No value is returned.
\end{description}
