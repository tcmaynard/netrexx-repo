\subsection{NetRexx arithmetic}\label{refoarith}
\index{Overview,Arithmetic}
\index{Arithmetic,overview}
:p.
Character strings in NetRexx are commonly used for arithmetic (assuming,
of course, that they represent numbers).  The string representation of
numbers can include integers, decimal notation, and exponential
notation; they are all treated the same way.  Here are a few:
\begin{verbatim}
'1234'
'12.03'
'-12'
'120e+7'
\end{verbatim}
:p.
The arithmetic operations in NetRexx are designed for people rather than
machines, so are decimal rather than binary, do not overflow at certain
values, and follow the rules that people use for arithmetic.
The operations are completely defined by the ANSI X3.274 standard for
Rexx, so correct implementations always give the same results.
:p.
An unusual feature of NetRexx arithmetic is the \texttt{numeric}
instruction: this may be used to select the \emph{arbitrary
precision} of calculations.  You may calculate to whatever
precision that you wish (for financial calculations, perhaps), limited
only by available memory.  For example:
\begin{verbatim}
numeric digits 50
say 1/7
\end{verbatim}
:p.
which would display
\begin{verbatim}
0.14285714285714285714285714285714285714285714285714
\end{verbatim}
:p.
The numeric precision can be set for an entire program, or be adjusted
at will within the program.  The \texttt{numeric}  instruction can also
be used to select the notation (\emph{scientific}
or \emph{engineering}) used for numbers in exponential format.
:p.
NetRexx also provides simple access to the native binary arithmetic of
computers.  Using binary arithmetic offers many opportunities for
errors, but is useful when performance is paramount.  You select binary
arithmetic by adding the instruction:
\begin{verbatim}
options binary
\end{verbatim}
:p.
at the top of a NetRexx program.  The language processor will then use
:a id=refobinar.binary arithmetic:ea. instead of NetRexx decimal
arithmetic for calculations, if it can, throughout the program.
