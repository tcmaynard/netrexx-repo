\chapter{Parsing templates}\label{refparsing}
\index{PARSE,parsing rules}
\index{Parsing,general rules}
\index{Templates, parsing,general rules}
 The \keyword{parse} instruction allows a selected string to
be parsed (split up) and assigned to variables, under the control of a
\emph{template}.
 
The various mechanisms in the template allow a string to be split up by
explicit matching of strings (called \emph{patterns}), or by
specifying numeric positions (\emph{positional patterns} - for
example, to extract data from particular columns of a line read from a
character stream).
Once split into parts, each segment of the string can then be assigned
to variables as a whole or by words (delimited by blanks).
 
This section first gives some informal examples of how the parsing
template can be used, and then defines the algorithms in detail.
\index{Parsing,introduction}
\section{Introduction to parsing}\label{parseintro}
 The simplest form of parsing template consists of a list of variable
names.
The string being parsed is split up into words (characters delimited by
blanks), and each word from the string is assigned to a
variable in sequence from left to right.
The final variable is treated specially in that it will be assigned
whatever is left of the original string and may therefore contain
several words.
For example, in the \keyword{parse} instruction:
\begin{alltt}
parse 'This is a sentence.' v1 v2 v3
\end{alltt}
the term (in this case a literal string) following the instruction
keyword is parsed, and then:  the variable \textbf{\emph{v1}}
would be assigned the value "\textbf{This}", \textbf{\emph{v2}}
would be assigned the value "\textbf{is}",
and \textbf{\emph{v3}} would be assigned the
value "\textbf{a sentence.}".
 
Leading blanks are removed from each word in the string before it is
assigned to a variable, as is the blank that delimits the end of the
word.
Thus, variables set in this manner (\textbf{\emph{v1}}
and \textbf{\emph{v2}} in the example) will never have leading or
trailing blanks, though \textbf{\emph{v3}} could have both leading
and trailing blanks.
 Note that the variables assigned values in a template are always
given a new value and so if there are fewer words in the string than
variables in the template then the unused variables will be set to the
null string.
 The second parsing mechanism uses a literal string in a template as a
pattern, to split up the string.
For example:
\begin{alltt}
parse 'To be, or not to be?' w1 ',' w2
\end{alltt}
would cause the string to be scanned for the comma, and then split
at that point; the variable \textbf{\emph{w1}} would be set
to "\textbf{To be}", and \textbf{\emph{w2}} is set to
"\textbf{ or not to be?}".
Note that the pattern itself (and \textbf{only} the pattern) is
removed from the string.
Each section of the string is treated in just the same way as the whole
string was in the previous example, and so either section could be split
up into words.
 Thus, in:
\begin{alltt}
parse 'To be, or not to be?' w1 ',' w2 w3 w4
\end{alltt}
\textbf{\emph{w2}} and \textbf{\emph{w3}} would be
assigned the values "\textbf{or}" and "\textbf{not}",
and \textbf{\emph{w4}} would be assigned the remainder:
"\textbf{to be?}".
 
If the string in the last example did not contain a comma, then
the pattern would effectively "match" the end of the string, so
the variable to the left of the pattern would get the entire input
string, and the variables to the right would be set to a null string.
 The pattern may be specified as a variable, by putting the variable
name in parentheses.  The following instructions therefore have the
same effect as the last example:
\begin{alltt}
c=','
parse 'To be, or not to be?' w1 (c) w2 w3 w4
\end{alltt}
 The third parsing mechanism is the numeric positional
pattern.
This works in the same way as the string pattern except that it
specifies a column number.  So:
\begin{alltt}
parse 'Flying pigs have wings' x1 5 x2
\end{alltt}
would split the string at the fifth column,
so \textbf{\emph{x1}} would be "\textbf{Flyi}"
and \textbf{\emph{x2}} would start at column 5 and
so be "\textbf{ng pigs have wings}".
 More than one pattern is allowed, so for example:
\begin{alltt}
parse 'Flying pigs have wings' x1 5 x2 10 x3
\end{alltt}
would split the string at columns 5 and 10,
so \textbf{\emph{x2}} would be
"\textbf{ng pi}" and \textbf{\emph{x3}} would be
"\textbf{gs have wings}".
 The numbers can be relative to the last number used, so:
\begin{alltt}
parse 'Flying pigs have wings' x1 5 x2 +5 x3
\end{alltt}
would have exactly the same effect as the last example; here
the \textbf{+5} may be thought of as specifying the length of the
string to be assigned to \textbf{\emph{x2}}.
 As with literal string patterns, the positional patterns can
be specified as a variable by putting the name of a variable, in
parentheses, in place of the number.
An absolute column number should then be indicated by using an equals
sign ("\textbf{=}") instead of a plus or minus sign.
The last example could therefore be written:
\begin{alltt}
start=5
length=5
data='Flying pigs have wings'
parse data  x1 =(start) x2 +(length) x3
\end{alltt}
 String patterns and positional patterns can be mixed (in effect the
beginning of a string pattern just specifies a variable column number)
and some very powerful things can be done with templates.
The next section describes in more detail how the various mechanisms
interact.
\index{Parsing,definition}
\section{Parsing definition}\label{}
 This section describes the rules that govern parsing.
\index{Parsing,general rules}
 In its most general form, a template consists of alternating pattern
specifications and variable names.  Blanks may be added between
patterns and variable names to separate the tokens and to improve
readability.  The patterns and variable names are used strictly in
sequence from left to right, and are used once only.  In practice,
various simpler forms are used in which either variable names or
patterns may be omitted; we can therefore have variable names without
patterns in between, and patterns without intervening variable names.
 In general, the value assigned to a variable is that sequence of
characters in the input string between the point that is matched by the
pattern on its left and the point that is matched by the pattern on its
right.
 If the first item in a template is a variable, then there is an
implicit pattern on the left that matches the start of the string, and
similarly if the last item in a template is a variable then there is an
implicit pattern on the right that matches the end of the string.
Hence the simplest template consists of a single variable name which in
this case is assigned the entire input string.
 Setting a variable during parsing is identical in effect to setting a
variable in an assignment.
 The constructs that may appear as patterns fall into two categories;
patterns that act by searching for a matching string (literal
patterns), and numeric patterns that specify an absolute or relative
position in the string (positional patterns).
Either of these can be specified explicitly in the template, or
alternatively by a reference to a variable whose value is to be used
as the pattern.
 For the following examples, assume that the following sample string
is being parsed; note that all blanks are significant - there are
two blanks after the first word "\textbf{is}" and also after the
second comma:
\begin{alltt}
'This is  the text which, I think,  is scanned.'
\end{alltt}
\subsection{Parsing with literal patterns}
\index{,}
\index{,}
\index{Parsing,literal patterns}
\index{Literal patterns,}
 Literal patterns cause scanning of the data string to find a
sequence that matches the value of the literal.  Literals are expressed
as a quoted string.  The null string matches the end of the data.
 The template:
\begin{alltt}
w1 ',' w2 ',' w3
\end{alltt}
when parsing the sample string, results in:
\begin{alltt}
\emph{w1} \emph{has the value} "This is  the text which"
\emph{w2} \emph{has the value} " I think"
\emph{w3} \emph{has the value} "  is scanned."
\end{alltt}
 Here the string is parsed using a template that asks that each of
the variables receive a value corresponding to a portion of the
original string between commas; the commas are given as quoted strings.
Note that the patterns themselves are removed from the data being
parsed.
 A different parse would result with the template:
\begin{alltt}
w1 ',' w2 ',' w3 ',' w4
\end{alltt}
which would result in:
\begin{alltt}
\emph{w1} \emph{has the value} "This is  the text which"
\emph{w2} \emph{has the value} " I think"
\emph{w3} \emph{has the value} "  is scanned."
\emph{w4} \emph{has the value} ""  (null string)
\end{alltt}
 This illustrates an important rule.  When a match for a pattern
cannot be found in the input string, it instead "matches" the end
of the string.  Thus, no match was found for the third \textbf{','} in
the template, and so \textbf{\emph{w3}} was assigned the rest of
the string. \textbf{\emph{w4}} was assigned a null string
because the pattern on its left had already reached the end of the
string.
 Note that \textbf{all} variables that appear in a template in this
way are assigned a new value.
\subsection{Parsing strings into words}
\index{Words,in parsing}
\index{Parsing,selecting words}
 If a variable is directly followed by one or more other variables,
then the string selected by the patterns is assigned to the variables
in the following manner.
Each blank-delimited word in the string is
assigned to each variable in turn, except for the last variable in the
group (which is assigned the remainder of the string).
The values of the variables which are assigned words will have neither
leading nor trailing blanks.
 Thus the template:
\begin{alltt}
w1 w2 w3 w4 ','
\end{alltt}
would result in:
\begin{alltt}
\emph{w1} \emph{has the value} "This'
\emph{w2} \emph{has the value} "is"
\emph{w3} \emph{has the value} "the"
\emph{w4} \emph{has the value} "text which"
\end{alltt}
Note that the final variable (\textbf{\emph{w4}} in this
example) could have had both leading blanks and trailing blanks, since
only the blank that delimits the previous word is removed from the data.
 Also observe that this example is not the same as specifying
explicit blanks as patterns, as the template:
\begin{alltt}
w1 ' ' w2 ' ' w3 ' ' w4 ','
\end{alltt}
would in fact result in:
\begin{alltt}
\emph{w1} \emph{has the value} "This'
\emph{w2} \emph{has the value} "is"
\emph{w3} \emph{has the value} ""  (null string)
\emph{w4} \emph{has the value} "the text which"
\end{alltt}
since the third pattern would match the third blank in the data.
 In general, when a variable is followed by another variable then
parsing of the input into individual words is implied.
The parsing process may be thought of as first splitting the original
string up into other strings using the various kinds of patterns, and
then assigning each of these new strings to (zero or more) variables.
\subsection{Use of the period as a placeholder}\label{refplaceh}
\index{Period,as placeholder in parsing}
\index{. (period),as placeholder in parsing}
 A period (separated from any symbols by at least one blank) acts as a
placeholder in a template.
It has exactly the same effect as a variable name, except that no
variable is set.
It is especially useful as a "dummy variable" in a list of
variables, or to collect (ignore) unwanted information at the end of a
string.  Thus the template:
\begin{alltt}
 . . . word4 .
\end{alltt}
would extract the fourth word ("\textbf{text}") from the sample
string and place it in the variable \textbf{\emph{word4}}.
Blanks between successive periods in templates may be omitted, so the
template:
\begin{alltt}
 ... word4 .
\end{alltt}
would have the same result as the last template.
\subsection{Parsing with positional patterns}
\index{Parsing,positional patterns}
\index{Positional patterns,}
 Positional patterns may be used to cause the parsing to occur on the
basis of position within the string, rather than on its contents.
\index{Signs in parsing templates,}
They take the form of whole numbers, optionally preceded by a plus,
minus, or equals sign which indicate relative or absolute positioning.
These may cause the matching operation to "back up" to an earlier
position in the data string, which can only occur when positional
patterns are used.
\index{Absolute,column specification in parsing}
\index{Column specification in parsing,}
 \textbf{Absolute positional patterns:}
A number in a template that is \textbf{not} preceded by a sign
refers to a particular (absolute)
character column in the input, with 1 referring to the first
column.
For example, the template:
\begin{alltt}
s1 10 s2 20 s3
\end{alltt}
results in:
\begin{alltt}
\emph{s1} \emph{has the value} "This is  "
\emph{s2} \emph{has the value} "the text w"
\emph{s3} \emph{has the value} "hich, I think,  is scanned."
\end{alltt}
 Here \textbf{\emph{s1}} is assigned characters from the first
through the ninth character, and \textbf{\emph{s2}} receives input
characters 10 through 19.
As usual the final variable, \textbf{\emph{s3}}, is assigned the
remainder of the input.
 
\index{= equals sign,in parsing template}
\index{Parsing,absolute columns}
An equals sign ("\textbf{=}") may be placed before the number
to indicate explicitly that it is to be used as an absolute column
position; the last template could have been written:
\begin{alltt}
s1 =10 s2 =20 s3
\end{alltt}
 A positional pattern that has no sign or is preceded by the
equals sign is known as an \emph{absolute positional pattern}.
\index{Absolute,positional pattern}
\index{Relative column specification in parsing,}
\index{+ plus sign,in parsing template}
\index{- minus sign,in parsing template}
 \textbf{Relative positional patterns:}
A number in a template that is preceded by a plus or minus sign
indicates
movement relative to the character position at which the previous
pattern match occurred.
This is a \emph{relative positional pattern}.
\index{Relative positional pattern,}
 If a plus or minus
is specified, then the position used for the next
match is calculated by adding (or subtracting) the number given to the
last matched position.
The last matched position is the position of the first character of the
last match, whether specified numerically or by a string.
 
For example, the instructions:
\begin{alltt}
parse '123456789'  3 w1 +3 w2 3 w3
\end{alltt}
result in
\begin{alltt}
\emph{w1} \emph{has the value} "345"
\emph{w2} \emph{has the value} "6789"
\emph{w3} \emph{has the value} "3456789"
\end{alltt}
The \textbf{+3} in this case is equivalent to the absolute
number \textbf{6} in the same position, and may also be considered to
be specifying the length of the data string to be assigned to the
variable \textbf{\emph{w1}}.
 This example also illustrates the effects of a positional pattern
that implies movement to a character position to the left of (or to)
the point at which the last match occurred.
The variable on the left is assigned characters through the end of the
input, and the variable on the right is, as usual, assigned characters
starting at the position dictated by the pattern.
 A useful effect of this is that multiple assignments can be made:
\begin{alltt}
parse x 1 w1 1 w2 1 w3
\end{alltt}
This results in assigning the (entire) value
of \textbf{\emph{x}}
to \textbf{\emph{w1}}, \textbf{\emph{w2}},
and \textbf{\emph{w3}}.
(The first "\textbf{1}" here could be omitted as it is
effectively the same as the implicit starting pattern described at the
beginning of this section.)
 If a positional pattern specifies a column that is greater than the
length of the data, it is equivalent to specifying the end of the data
(\emph{i.e.}, no padding takes place).
Similarly, if a pattern specifies a column to the left of the first
column of the data, this is not an error but instead is taken to
specify the first column of the data.
 Any pattern match sets the "last position" in a string to which
a relative positional pattern can refer.
The "last position" set by a literal pattern is the position at
which the match occurred, that is, the position in the data of the
\emph{first} character in the pattern.
The literal pattern in this case is \textbf{not} removed from the
parsed data.
Thus the template:
\begin{alltt}
',' -1 x +1
\end{alltt}
 will:
\begin{enumerate}
\item Find the first comma in the input (or the end of the string if
there is no comma).
\item Back up one position.
\item Assign one character (the character immediately preceding the comma
or end of string) to the variable \textbf{\emph{x}}.
\end{enumerate}
 One possible application of this is looking for abbreviations in a
string. Thus the instruction:
\begin{alltt}
/* Ensure options have a leading blank and are
   in uppercase before parsing. */
parse (' 'opts).upper ' PR' +1 prword ' '
\end{alltt}
will set the variable \textbf{\emph{prword}} to the first word
in \textbf{\emph{opts}} that starts with "\textbf{PR}" (in
any case), or will set it to the null string if no such word exists.
 \textbf{Notes:}
\begin{enumerate}
\item The positional patterns \textbf{+0} and \textbf{-0} are valid,
have the same effect, and may be used to include the whole of a previous
literal (or variable) pattern within the data string to be parsed into
any following variables.
\item As illustrated in the last example, patterns may follow each other
in the template without intervening variable names.  In this case each
pattern is obeyed in turn from left to right, as usual.
\item There may be blanks between the sign in a positional pattern and
the number, because \nr{} defines that blanks adjacent to special
characters are removed.
\end{enumerate}
\subsection{Parsing with variable patterns}
\index{Variables,in parsing patterns}
\index{Parsing,variable patterns}
\index{Parentheses,in parsing templates}
 It is sometimes desirable to be able to specify a pattern by using
the value of a variable instead of a fixed string or number.
This may be achieved by placing the name of the variable to be used as
the pattern in parentheses (blanks are not necessary either inside or
outside the parentheses, but may be added if desired).
\index{Variable reference,in parsing template}
This is called a \emph{variable reference}; the value of the variable
is converted to string before use, if necessary.
 If the parenthesis to the left of the variable name is not preceded
by an equals, plus, or minus sign ("\textbf{=}",
"\textbf{+}", or "\textbf{-}")
the value of the variable is then used as though it were a literal
(string) pattern.
The variable may be one that has been set earlier in the parsing
process, so for example:
\begin{alltt}
input="L/look for/1 10"
parse input  verb 2 delim +1 string (delim) rest
\end{alltt}
will set:
\begin{alltt}
\emph{verb} \emph{to} 'L'
\emph{delim} \emph{to} '/'
\emph{string} \emph{to} 'look for'
\emph{rest} \emph{to} '1 10'
\end{alltt}
 If the left parenthesis \textbf{is} preceded by an equals, plus,
or minus sign then the value of the variable is used as an absolute or
relative positional pattern (instead of as a literal string pattern).
In this case the value of the variable must be a non-negative
whole number, and (as before) it may have been set earlier in the
parsing process.
\index{,}
\index{,}
\index{,}
\index{,}
