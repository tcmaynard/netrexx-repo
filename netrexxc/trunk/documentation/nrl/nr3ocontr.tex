\subsection{Control instructions}\label{refocontr}
\index{Overview,control instructions}
\index{Control instructions, overview,}
:p.
NetRexx provides a selection of \emph{control} instructions, whose form was
chosen for readability and similarity to natural languages.  The control
instructions include \texttt{if}... \texttt{then}... \texttt{else} (as
in the "greet" example) for simple conditional processing:
\begin{verbatim}
if ask='Yes' then say "You answered Yes"
             else say "You didn't answer Yes"
\end{verbatim}
:p.
\texttt{select}... \texttt{when}... \texttt{otherwise}... \texttt{end}
for selecting from a number of alternatives:
\begin{verbatim}
select
  when a>0 then say 'greater than zero'
  when a>0 then say 'less than zero'
  otherwise say 'zero'
  end

select case i+1
  when 1 then say 'one'
  when 1+1 then say 'two'
  when 3, 4, 5 then say 'many'
  end
\end{verbatim}
:p.
\texttt{do}... \texttt{end} for grouping:
\begin{verbatim}
if a>3 then do
  say 'A is greater than 3; it will be set to zero'
  a=0
  end
\end{verbatim}
:p.
and \texttt{loop}... \texttt{end} for repetition:
\begin{verbatim}
/* repeat 10 times; I changes from 1 to 10 */
loop i=1 to 10
  say i
  end i
\end{verbatim}
:p.
The \texttt{loop} instruction can be used to step a variable \texttt{to}
some limit, \texttt{by} some increment, \texttt{for} a specified number of
iterations, and \texttt{while} or \texttt{until} some condition is
satisfied.  \texttt{loop forever} is also provided, and
\texttt{loop over} can be used to work through a collection of
variables.
:p.
Loop execution may be modified by \texttt{leave} and \texttt{iterate}
instructions that significantly reduce the complexity of many programs.
:p.
\index{Exceptions,overview}
The \texttt{select}, \texttt{do}, and \texttt{loop} constructs also have
the ability to :a id=refoexcep."catch" exceptions:ea. that occur in
the body of the construct.  All three, too, can specify a
\texttt{finally} instruction which introduces instructions which are to
be executed when control leaves the construct, regardless of how the
construct is ended.
