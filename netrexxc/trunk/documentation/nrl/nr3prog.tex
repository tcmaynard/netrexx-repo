\chapter{Program structure}\label{refpstruct}
\index{Programs,}
\index{Programs,structure}
\index{Program,structure}
 A \nr{} \emph{program} is a collection of
 clauses (see page \pageref{refclau})  derived from a single implementation-defined
source stream (such as a file).
When a program is processed by a language processor
\footnote{
Such as a compiler or interpreter.
}
it defines one or more classes.
Classes are usually introduced by the  \keyword{class} instruction (see page \pageref{refclass}), but if the first is a standard class, intended to be
run as a stand-alone application, then the \keyword{class} instruction
can be omitted.  In this case, \nr{} defines an implied class
and initialization method that will be used.
 
The implied class and method permits the writing of "low
boilerplate" programs, with a minimum of syntax.
The simplest, documented, \nr{} program that has an effect might
therefore be:
 
\textbf{Example:}
\index{Example,Hello World}
\begin{lstlisting}[label=hello,caption=hello.nrx]
/* This is a very simple NetRexx program */
say 'Hello World!'
\end{lstlisting}
 
In more detail, a \nr{} program consists of:
\index{Program,prolog}
\index{Prolog, of a program,}
\begin{enumerate}
\item An optional \emph{prolog} (\keyword{package}, \keyword{import}, and
\keyword{options} instructions).  Only one \keyword{package} instruction
is permitted per program.
\item  One or more class definitions, each introduced by a \keyword{class}
instruction.
\end{enumerate}
 
\index{Class,definition}
A \emph{class definition} comprises:
\begin{enumerate}
\item The \keyword{class} instruction which introduces the class (which may
be inferred, see below).
\item 
Zero or more property variable assignments,
\index{Properties,initialization}
\index{Assignment,property initialization}
along with optional \keyword{properties}
instructions that can alter their attributes, and optional
\keyword{numeric} and \keyword{trace} instructions.
Property variable assignments take the form of an
 \emph{assignment} (see page \pageref{refassign}) , with an optional
"\textbf{=}" and expression, which may:
\begin{itemize}
\item just name a property (by omitting the "\textbf{=}"
and expression of the assignment), in which case it refers to a string of
type \textbf{R\textsc{exx}}
\item assign a type to the property (when the expression evaluates to just
a type)
\item 
assign a type and initial value to the property (when the expression
returns a value).
\end{itemize}
\item Zero or more method definitions, each introduced by a
\keyword{method} instruction (which may be inferred if the \keyword{class}
instruction is inferred, see below).
\end{enumerate}
 
\index{Method,definition}
A \emph{method definition} comprises:
\begin{itemize}
\item 
Any \nr{} instructions, except the \keyword{class}, \keyword{method},
and \keyword{properties} instructions and those allowed in the prolog
(the \keyword{package}, \keyword{import}, and \keyword{options}
instructions).
\end{itemize}
 \textbf{Example:}
\index{Example,of two classes}
\begin{lstlisting}[label=testclass,caption=testclass.nrx]
/* A program with two classes */
import java.applet.   -- for example

class testclass extends Applet
  properties public
    state             -- property of type 'Rexx'
    i=int             -- property of type 'int'
  properties constant
    j=int 3           -- property initialized to '3'

  method start
    say 'I started'
    state='start'

  method stop
    say 'I stopped'
    state='stop'

class anotherclass
  method testing
    loop i=1 to 10
      say '1, 2, 3, 4...'
      if i=7 then return
     end
    return

  method anothertest
    say '1, 2, 3, 4'
\end{lstlisting}
This example shows a prolog (with just an \keyword{import}
instruction) followed by two classes.  The first class includes
two public properties, one constant property, and two methods.
The second class includes no properties, but also has two methods.
 
Note that a \keyword{return} instruction implies no static scoping; the
content of a method is ended by a \keyword{method} (or \keyword{class})
instruction, or by the end of the source stream.
The \keyword{return} instruction at the end of the \textbf{testing}
method is, therefore, unnecessary.
\section{Program defaults}\label{programdefaults}
 
The following defaults are provided for \nr{} programs:
\begin{enumerate}
\item If, while parsing prolog instructions, some instruction that is not
valid for the prolog and is not a \keyword{class} instruction is
encountered, then a default \keyword{class} instruction (with an
implementation-provided short name, typically derived from the name of
the source stream) is inserted.  If the instruction was not a
\keyword{method} instruction, then a default \keyword{method} instruction
(with a name and attributes appropriate for the environment, such
as \textbf{main}) is also inserted.
 
In this latter case, it is assumed that execution of the program will
begin by invocation of the default method.
In other words, a "stand-alone" application can be written without
explicitly providing the class and method instructions for the first
method to be executed.
An example of such a program is given in  Appendix A (see page \pageref{refappa}) .

 
\emph{In the reference implementation, the \textbf{main} method in a
stand-alone application is passed the words forming the command string
as an array of strings of type \textbf{java.lang.String} (one word to
each element of the array).
When the \nr{} reference implementation provides the \textbf{main}
method instruction by default, it also constructs a \nr{} string of
type \textbf{R\textsc{exx}} from this array of words, with a blank added
between words, and assigns the string to the variable
\textbf{arg}.}
 
\emph{The command string may also have been edited by the underlying
operating system environment; certain characters may not be
allowed, multiple blanks or whitespace may have been reduced to
single blanks, etc.
}
\item If a method ends and the last instruction at the outer level of the
method scope is not \keyword{return} then a \keyword{return} instruction
is added if it could be reached.
In this case, if a value is expected to be returned by the method (due
to other \keyword{return} instructions returning values, or there being a
\keyword{returns} keyword on the \keyword{method} instruction), an error
is reported.
\end{enumerate}
 
Language processors may provide options to prevent, or warn of, these
defaults being applied, as desired.
