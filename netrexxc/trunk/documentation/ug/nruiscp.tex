% .* ------------------------------------------------------------------
% .* \nr{} User's Guide                                              mfc
% .* Copyright (c) IBM Corporation 1996, 2000.  All Rights Reserved.
% .* ------------------------------------------------------------------
\chapter{Setting the CLASSPATH}
\index{CLASSPATH, setting}
\index{setting CLASSPATH}
Most implementations of Java use an \emph{environment variable} called
CLASSPATH to indicate a search path for Java classes.  The Java Virtual
Machine and the \nr{} translator rely on the CLASSPATH value to find
directories, zip files, and jar files which may contain Java classes.
\newline
The procedure for setting the CLASSPATH environment variable depends on
your operating system (and there may be more than one way).  Here are
some examples:
:ul.
:li.
For most :b.Windows:eb. installations, or for :b.OS/2:eb., use a :m.SET
CLASSPATH=:em. command in AUTOEXEC.BAT (for Windows) or in CONFIG.SYS
(for OS/2), and then re-boot after changing.  In both cases the command
syntax is the same, and might look like this:
\begin{verbatim}
set classpath=.;c:\java1.2\lib\NetRexxC.jar
\end{verbatim}
In this example, the first segment of the value (before the semicolon)
lets classes in the current directory be found, and the second segment
includes the classes needed by the \nr{} translator.  Both
environments normally include the standard Java classes automatically.
Under Java 1.2, you may need to add the Sun tools classes explicitly (in
tools.jar, see above).
.* - - - -
:li.
Under :b.Windows NT 4.0:eb. and :b.Windows 2000:eb. the CLASSPATH should
be set using Start, Settings, Control Panel, System, Environment tab,
System Variables, and clicking on CLASSPATH; new command windows will
then inherit the new setting immediately.
.* - - - -
:li.
For :b.Linux:eb. and :b.Unix:eb. (:b.BASH:eb., :b.Korn:eb., or
:b.Bourne:eb. shell), use:
\begin{verbatim}
CLASSPATH=<newdir>:\$CLASSPATH
export CLASSPATH
\end{verbatim}
Changes for re-boot or opening of a new window should be placed
in your :m./etc/profile:em., :m..login:em., or :m..profile:em. file, as
appropriate.
.* - - - -
:li.
For :b.Linux:eb. and :b.Unix:eb.
(:b.C:eb. shell), use:
\begin{verbatim}
setenv CLASSPATH <newdir>:\$CLASSPATH
\end{verbatim}
Changes for re-boot or opening of a new window should be
placed in your :m..cshrc:em. file.
:eul.
\newline
If you are unsure of how to do this, check the documentation you have
for installing the Java toolkit.
