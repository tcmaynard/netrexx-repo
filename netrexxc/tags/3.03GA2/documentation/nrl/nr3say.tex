\chapter{Say instruction}\label{refsay}
\index{SAY,instruction}
\index{Instructions,SAY}
\index{,}
\index{,}
\index{Console, writing to with SAY,}
\index{Terminal, writing to with SAY,}
\index{stdout, writing to with SAY,}
\begin{shaded}
\begin{alltt}
\textbf{say} [\emph{expression}];
\end{alltt}
\end{shaded}
 \keyword{say} writes a string to the default output character
stream.
This typically causes it to be displayed (or spoken, or typed, \emph{etc.}) to
the user.

\textbf{Example:}
\begin{alltt}
data=100
say data 'divided by 4 =>' data/4
/* would display:  "100 divided by 4 => 25"  */
\end{alltt}
 
The result of evaluating the \emph{expression} is expected to be a
string; if it is not a string, it will be converted to a string.
This result string is written from the program via an
implementation-defined output stream.
 
\index{Line, displaying,}
\begin{shaded}\noindent
By default, the result string is treated as a "line" (an
implementation-dependent mechanism for indicating line termination is
effected after the string is written).
If, however, the string ends in the NUL character
(\textbf{'\textbackslash -'} or \textbf{'\textbackslash 0'}) then that character
is removed and line termination is not indicated.
\end{shaded}\indent
The result string may be of any length.  If no expression is specified,
or the expression result is \textbf{null}, then an empty line is
written (that is, as though the expression resulted in a null string).
