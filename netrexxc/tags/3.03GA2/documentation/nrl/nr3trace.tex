\chapter{Trace instruction}\label{reftrace}
\index{TRACE,instruction}
\index{Trace setting,altering with TRACE instruction}
\index{Instructions,TRACE}
\index{Tracing,execution of programs}
\index{,}
\index{,}
\begin{shaded}
\begin{alltt}
\textbf{trace} \emph{traceoption};

where \emph{traceoption} is one of:
    \emph{tracesetting}
    \textbf{var} [\emph{varlist}]

where \emph{tracesetting} is one of:

    \textbf{all}
    \textbf{methods}
    \textbf{off}
    \textbf{results}

and \emph{varlist} is one or more variable \emph{names}, optionally prefixed with a \texttt{+} or \texttt{-}
\end{alltt}
\end{shaded}
 The \keyword{trace} instruction is used to control the tracing of the
execution of \nr{} methods, and is primarily used for debugging.
It may change either the general trace \emph{setting} or may select
or deselect the tracing of individual variables.
 
Within methods, the \keyword{trace} instruction changes the trace setting
or variables tracing when it is executed, and affects the tracing of
all clauses in the method which are then executed (until changed by a
later \keyword{trace} instruction).
 
One or more \keyword{trace} instructions may appear before the first
method in a class, one of which may set the initial trace setting
for all methods in the class (the default is \keyword{off}) and others
may set up variables tracing that applies to all the methods in the
class.
These act as though the \keyword{trace} instructions were
placed immediately following the \keyword{method} instruction in each
method (except that they will not be traced).
 
Similarly, one or more \keyword{trace} instructions may be placed
before the first \keyword{class} instruction in a program; they do not
imply the start of a class.  One of these may set the initial trace
setting and others may set up variables tracing for all classes in
the program (except interface classes) and act as though the
\keyword{trace} instructions were placed immediately following the
\keyword{class} instruction in each class.
\section{Tracing clauses}
\index{Trace setting,}
\index{Tracing,clauses}
 
The trace \emph{setting} controls the tracing of clauses in a program, and
may be one of the following:
\begin{description}
\item[all]\label{reftrall}
\index{ALL,TRACE setting}
 All clauses (except null clauses without commentary) which are in
methods and which are executed after the \keyword{trace} instruction will
be traced.
If \keyword{trace all} is placed before the first method in the current
class, the \keyword{method} instructions in the class,
together with the values of the arguments passed to each method,
will be traced when the method is invoked (that is, \keyword{trace all}
implies \keyword{trace methods}).
\item[methods]\label{reftrmeth}
\index{METHODS,TRACE setting}
 
All \keyword{method} clauses in the class will be traced when the method
they introduce is invoked, together with the values of the arguments
passed to each method; no other clauses, or results, will be traced.
The \keyword{trace methods} instruction must be placed before the first
method in the current class (as otherwise it would have no effect).
\item[off]\label{reftroff}
\index{OFF,TRACE setting}
 
Turns tracing off; no following clauses, variables, or results will be traced.
\item[results]\label{reftrres}
\index{RESULTS,TRACE setting}
 All clauses (except null clauses without commentary) which are in
methods and which are executed after the \keyword{trace} instruction will
be traced, as though \keyword{trace all} had been requested.
In addition, the results of all \emph{expression} evaluations and
any results assigned to a variable by an assignment, \keyword{loop}, or
\keyword{parse} instruction are also traced.
 
If \keyword{trace results} is placed before the first method in the
current class, the \keyword{method} instructions in the class will be
traced when the method is invoked, together with the values of the
arguments passed to each method.
\end{description}
 \textbf{Notes:}
\begin{enumerate}
\item Tracing of clauses shows the data from the source of the program,
starting at the first character of the first token of the clause and
including any commentary from that point until the end of the clause.
\item When a loop is being traced, the \keyword{loop} clause itself will be
traced on every iteration of the loop, as indicated by the
 programmer's model (see page \pageref{refloopmod}) ; the \keyword{end} clause is only
traced once, when the loop completes normally.
\item With \keyword{trace results}, an expression is not traced if it is
immediately used for an assignment (in an assignment instruction, or
when the control variable is initialized in a \keyword{loop}
instruction).
The assignment will trace the result of the expression.
\end{enumerate}
\section{Tracing variables}
\index{Tracing,variables}
 
The \keyword{var} option adds names to a list of monitored
variables; it can also remove names from the list.  If the name of a
variable in the current class or method is in the list, then \keyword{trace
results} is turned on for any assignment, \keyword{loop}, or
\keyword{parse} clause that assigns a new value to the named
variable.
 
Variable names are specified by listing them after the \keyword{var}
keyword.
Each name may be optionally prefixed by a \texttt{+} or a \texttt{-} sign.
A \texttt{+} sign indicates that the variable is to be added to the list
of monitored variables (the default), and a \texttt{-} sign indicates that
the variable is to be removed from the list.  Blanks may be added before
and after variable names and signs to separate the tokens and to improve
readability.
 For example:
\begin{alltt}
trace var a b c
-- now variables a, b, and c will be traced
trace var -b -c d
-- now variables a and d will be traced
\end{alltt}
 \textbf{Notes:}
\begin{enumerate}
\item 
Names in the list following the \keyword{var} keyword are simple symbols
that name variables in the current class or current method.
The variables may be properties, method arguments, or local variables,
and may be of any type, including arrays.
The names are not case-sensitive; any variables whose names match,
independent of case, will be monitored.
\item 
No variable name can appear more than once in the list on one
\keyword{trace var} instruction.  However, it is not an error to add the
name of a variable which does not exist or is not then assigned a value.
Similarly, it is not an error to remove a name which is not currently
being monitored.
\item 
One or more \keyword{trace var} instructions (along with one other
\keyword{trace} instruction) are allowed before the first method in a
class.  They all modify an initial list of monitored variables which
is then used for all methods in the class.  Similarly, \keyword{trace
var} instructions are allowed before the first class in a program,
in which case they apply to all classes (except interface classes).
\item 
Other \keyword{trace} instructions do not affect the list of monitored
variables.
The \keyword{trace off} instruction may be used to turn off tracing
completely; in this case \keyword{trace var} (with or without any
variable names) will then turn the tracing of variables back on, using
the current (or modified) variable list.
\item 
For a \keyword{parse} instruction, only monitored variables have their
assignments traced (unless \keyword{trace results} is already in effect).
\end{enumerate}
\section{The format of trace output}
\index{Indention during tracing,}
\index{Formatting,of output during tracing}
 
Trace output is either clauses from the program being traced, or results
(such as the results from expressions).
 
\index{Tracing,line numbers}
\index{Line numbers, in tracing,}
The first clause or result traced on any line will be preceded by its
line number in the program; this is right-justified in a space which
allows for the largest line number in the program, plus one blank.
Following clauses or results from the same line are preceded by white
space of the same width; however, any change of line number causes the
line number to be included.
 
Clauses that are traced will be displayed with the formatting
(indention) and layout used in the original source stream for the
program, starting with the first character of the first token of the
clause.
 
Results (if requested) are converted to a string for tracing if
necessary, are not indented, and have a double quote prefixed and
suffixed so that leading and trailing blanks are apparent; if, however,
the result being traced is  \textbf{null} (see page \pageref{refswnull})  then the
string "\textbf{[null]}" is shown (without quotes).
For results with an associated name (the values assigned to local
variables, method arguments, or properties in the current class), the
name of the result precedes the data, separated by a single blank.
 
For clarity, implementations may replace "control codes"
in the encoding of results (for example, EBCDIC values less
than \textbf{'\textbackslash x40'}, or Unicode values less than
\textbf{'\textbackslash x20'})
by a question mark ("\textbf{?}").
\index{Tracing,data identifiers}
 All lines displayed during tracing have a three character tag to
identify the type of data being traced.  This tag follows the line
number (or the space for a line number), and is separated from the line
number by a single blank.
The traced clause or result follows the tag, after another
blank.
The identifier tags are listed in table \ref{table:Trace
    Identifier Tags}.
\begin{table}\caption{Trace identifier tags}\label{table:Trace
    Identifier Tags}
\begin{tabularx}{\textwidth}{>{\bfseries}lX}
\toprule
\texttt{*=*}&
\index{*=* tracing flag,}
identifies the first line of the source of a single clause, \emph{i.e.},
the data actually in the program.
\\\midrule
\texttt{*-*}&
\index{*-* tracing flag,}
identifies a continuation line from the source of a single clause.
Continuations may be due to the use of a  continuation character (see page \pageref{refsemis}) 
or to the use of a  block comment (see page \pageref{refblockco}) 
which spans more than one line.
\\\midrule
\texttt{>a>}&
\index{>a> tracing flag,}
Identifies a value assigned to a method argument of the current
method.
The name of the argument is included in the trace.
\\\midrule
\texttt{>p>}&
\index{>p> tracing flag,}
Identifies a value assigned to a property.
The name of the property is included in the trace if the property is in
the current class.
\\\midrule
\texttt{>v>}&
\index{>v> tracing flag,}
Identifies a value assigned to a local variable in the current
method.
The name of the variable is included in the trace.
\\\midrule
\texttt{>>>}&
\index{>>> tracing flag,}
Identifies the result of an expression evaluation that is not used
for an assignment (for example, an argument expression in a method
call).
\\\midrule
\texttt{+++}&
\index{+++ tracing flag,}
Reserved for error messages that are not supplied by the environment
underlying the implementation.
\\\bottomrule
\end{tabularx}
\end{table}


\index{Trace,context}
\index{Thread,tracing}
If a trace line is produced in a different context (program or thread)
from the preceding trace line (if any) then a \emph{trace context}
line is shown.  This shows the name of the program that produced the
trace line, and also the name of the thread (and thread group) of the
context.
 
The thread group name is not shown if it is \texttt{main}, and in this
case the thread name is then also suppressed if its name is \texttt{main}.
 
\textbf{Examples:}
 If the following instructions, starting on line 53 of a 120-line
program, were executed:
\begin{alltt}
trace all
if i=1 then say 'Hello'
       else say 'i<>1'
say -
 'A continued line'
\end{alltt}
the trace output (if \emph{i} were \textbf{1}) would be:
\begin{alltt}
  54 *=* if i=1
     *=*        then
     *=*             say 'Hello'
  56 *=* say -
  57 *-*  'A continued line'
\end{alltt}
 Similarly, for the 3-line program:
\begin{alltt}
trace results
number=1/7
parse number before '.' after
\end{alltt}
the trace output would be:
\begin{alltt}
 2 *=* number=1/7
   >v> number "0.142857143"
 3 *=* parse number before '.' after
   >v> before "0"
   >v> after "142857143"
\end{alltt}
 \textbf{Notes:}
\begin{enumerate}
\item 
Trace output is written to an implementation-defined output stream
\index{stderr, used by TRACE,}
(typically the "standard error" output stream, which lets it be
redirected to a destination separate from the default destination for
output which is used by the \keyword{say} instruction).
\item In some implementations, the use of \keyword{trace} instructions
may substantially increase the size of classes and the execution time of
methods affected by tracing.
\footnote{
\emph{In the reference implementation, \keyword{options notrace} may be
used to disable all \keyword{trace} instructions and hence ensure that
tracing overhead is not accidentally incurred.}
}
\item With some implementations it may be possible to switch tracing on
externally, without requiring modification to the program.
\end{enumerate}
