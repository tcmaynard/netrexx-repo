\subsection{Extending classes}\label{refoexten}
\index{Overview,extending classes}
\index{Extending classes,overview}
\index{Classes,overview}
:p.
It's common, when dealing with objects, to take an existing class and
extend it.  One way to do this is to modify the source code of the
original class - but this isn't always available, and with many
different people modifying a class, classes could rapidly get
over-complicated.
:p.
Languages that deal with objects, like NetRexx, therefore allow new
classes of objects to be set up which are derived from existing classes.
For example, if you wanted a different kind of Oblong in which the
Oblong had a new property that would be used when printing the Oblong as
a rectangle, you might define it thus:
\index{Programs,examples}
\index{Example,program}
\begin{verbatim}
/* charOblong.nrx -- an oblong class with character */
class charOblong extends Oblong
  printchar       -- the character for display

  /* Constructor to make a new oblong with character */
  method charOblong(newwidth, newheight, newprintchar)
    super(newwidth, newheight) -- make an oblong
    printchar=newprintchar     -- and set the character

  /* 'Print' the oblong */
  method print
    loop for super.height
      say printchar.copies(super.width)
      end
\end{verbatim}
:p.
There are several things worth noting about this example:
:ol.
:li.
The ":m.extends Oblong:em." on the class instruction means that
this class is an extension of the Oblong class.  The properties and
methods of the Oblong class are \emph{inherited} by this class (that
is, appear as though they were part of this class).
:p.
Another common way of saying this is that ":m.charOblong:em."
is a \emph{subclass} of ":m.Oblong:em." (and
":m.Oblong:em." is the \emph{superclass} of
":m.charOblong:em.").
:li.
This class adds the :m.printchar:em. property to the properties
already defined for Oblong.
:li.
The constructor for this class takes a width and height (just like
Oblong) and adds a third argument to specify a print character.  It
first invokes the constructor of its superclass (Oblong) to build
an Oblong, and finally sets the printchar for the new object.
:li.
The new charOblong object also prints differently, as a rectangle
of characters, according to its dimension.  The :m.print:em. method
(as it has the same name and arguments - none - as that of the
superclass) replaces (overrides) the :m.print:em. method of Oblong.
:li.
The other methods of Oblong are not overridden, and therefore can
be used on charOblong objects.
:eol.
:p.
The :m.charOblong.nrx:em. file is compiled just
like :m.Oblong.nrx:em. was, and should create a file
called :m.charOblong.class:em..
:p.
Here's a program to try it out:
\begin{verbatim}
/* trycharOblong.nrx -- try the charOblong class */

first=charOblong(5,3,'#')  -- make an oblong
first.print                -- show it
first.relsize(1,1).print   -- enlarge and print again

second=charOblong(1,2,'*') -- make another oblong
second.print               -- and print it
\end{verbatim}
:p.
This should create the two charOblong objects, and print them out in a
simple "character graphics" form.
Note the use of the method :m.relsize:em. from Oblong to resize
the charOblong object.
:h4.Optional arguments
:p.
All methods in NetRexx may have optional arguments (omitted from the
right) if desired.  For an argument to be optional, you must supply a
default value.  For example, if the charOblong constructor was to have a
default value for :m.printchar:em., its method instruction could
have been written:
\begin{verbatim}
method charOblong(newwidth, newheight, newprintchar='X')
\end{verbatim}
:p.
which indicates that if no third argument is supplied
then :m.'X':em. should be used.  A program creating a charOblong
could then simply write:
\begin{verbatim}
first=charOblong(5,3)       -- make an oblong
\end{verbatim}
:p.
which would have exactly the same effect as if :m.'X':em. were
specified as the third argument.
