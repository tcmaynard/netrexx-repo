%%%%%% .* ------------------------------------------------------------------
% .* \nr{} User's Guide                                              mfc
% .* Copyright (c) IBM Corporation 1996, 2000.  All Rights Reserved.
% .* ------------------------------------------------------------------

\index{compiling, \nr{} programs}
\index{using the translator, as a Compiler}

\chapter{Using the translator}
\index{using the translator}
This section of the document tells you how to use the
translator package.  It assumes you have successfully installed Java and
\nr{}, and have tested that the \emph{hello.nrx} testcase can be
compiled and run, as described in the \emph{Testing the
\nr{} Installation} (section \ref{testing} on page \pageref{testing}).

The \nr{} translator may be used as a compiler or as an interpreter
(or it can do both in a single run, so parsing and syntax checking are
only carried out once).  It can also be used as simply a syntax checker.

When used as a compiler, the intermediate Java source code may be
retained, if desired.  Automatic formatting, and the inclusion of comments
from the \nr{} source code are also options.

\section{Using the translator as a compiler}
The installation instructions for the \nr{} translator describe how to
use the package to compile and run a simple \nr{} program
(\emph{hello.nrx}).  When using the translator in this way (as a
compiler), the translator parses and checks the \nr{} source code, and
if no errors were found then generates Java source code.  This Java code
(which is known to be correct) is then compiled into bytecodes
(\emph{.class} files) using a Java compiler.  By default,
the \emph{javac} compiler in the Java toolkit is used.

This section explains more of the options available to you when using
the translator as a compiler.
% .*
\section{The translator command}
\index{command, for compiling}

\index{\nr{}C, class}
The translator is invoked by running a Java program (class) which is
called 
\begin{verbatim}
org.netrexx.process.NetRexxC
\end{verbatim}  
(\textbf{\nr{}C}, for short). This can be run by using the Java interpreter, for example,
by the command:
\begin{verbatim}
java org.netrexx.process.NetRexxC
\end{verbatim}
\index{scripts, \nr{}C}
\index{\nr{}C, scripts}
\index{scripts, nrc}
\index{nrc scripts}
or by using a system-specific script (such as \emph{\nr{}C.cmd}.
or \emph{nrc.bat}).  In either case, the compiler invocation is followed
by one or more file specifications (these are the names of the files
containing the \nr{} source code for the programs to be compiled).

\index{file specifications}
File specifications may include a path; if no path is given then
\nr{}C will look in the current (working) directory for the file.
\nr{}C will add the extension \emph{.nrx} to input program names (file
specifications) if no extension was given.

So, for example, to compile \emph{hello.nrx} in the current directory,
you could use any of:
\begin{verbatim}
java org.netrexx.process.NetRexxC hello
java org.netrexx.process.NetRexxC hello.nrx
NetRexxC hello.nrx
nrc hello
\end{verbatim}
(the first two should always work, the last two require that the
system-specific script be available).  The resulting \emph{.class} file
is placed in the current directory, and the \emph{.crossref}
(cross-reference) file is placed in the same directory as the source
file (if there are any variables and the compilation has no errors).

Here's an example of compiling two programs, one of which is in the
directory \emph{d:\textbackslash myprograms}:
\begin{verbatim}
nrc hello d:\myprograms\test2.nrx
\end{verbatim}

In this case, again, the \emph{.class} file for each program is placed
in the current directory.

Note that when more than one program is specified, they are all compiled
within the same class context.  That is, they can see the
classes, properties, and methods of the other programs being compiled,
much as though they were all in one file.
\footnote{The programs do, however, maintain their independence (that is, they may
have different \textbf{options}, \textbf{import}, and \textbf{package}
instructions).}
This allows mutually interdependent programs and classes to be compiled
in a single operation.
Note that if you use the \textbf{package} instruction you should also
read the more detailed \emph{Compiling multiple
programs} section.% \ref{multiple} on page \pageref{multiple}.

\index{completion codes, from translator}
\index{return codes, from translator}
On completion, the \nr{}C class will exit with one of three return
values: 0 if the compilation of all programs was successful, 1 if there
were one or more Warnings, but no errors, and 2 if there were one or
more Errors. The result can be forced to 0 for warnings only with the
\emph{-warnexit0} option.

\index{option words}
\index{flags}
As well as file names, you can also specify various option words, which
are distinguished by the word being prefixed with \emph{-}.  These
flagged words (or flags) may be any of the option words allowed
on the \nr{} \textbf{options} instruction (see the \nr{} languagen
documentation, and the below paragraph).  These options words can be freely mixed with file
specifications.  To see a full list of options, execute the \nr{}C
command without specifying any files. As this command states, all options may have prefix 'no' added for the inverse effect.

\subsection{Options}
\index{compiling,options}
There are a number of options for the translator, some of which can be specified on the translator command line, and others also in the program source on the \textbf{option} statement. In the following table, c stands for \emph{commandline only}, s stands for \emph{source} and b stands for \emph{both}.
\begin{longtable}[l]{|l|p{10cm}|l|}
\caption{ Options } \\
\hline
\rowcolor[gray]{0.8} \bfseries Option & \bfseries Meaning & \bfseries Place   \
\endfirsthead
\multicolumn{3}{r}%
{{\tablename\ \thetable{} -- \emph{continued from previous page}}} \\
\endhead
\hline \multicolumn{3}{r}{\emph{Continued on next page}}
\endfoot

\endlastfoot
\rowcolor[gray]{0.8} \bfseries \huge   & \normalsize  &  \\
\hline
-arg words & interpret; remaining words are arguments & c \\
\hline
-binary &  classes are binary classes & b \\
\hline
 -compile  & compile (default; -nocompile implies -keep) & c \\
\hline
 -comments     & copy comments across to generated .java &b \\
\hline
 -compact      & display error messages in compact form &b \\
\hline
 -console   & display messages on console (default) &c \\
\hline
 -crossref     & generate cross-reference listing &b \\
\hline
 -decimal      & allow implicit decimal arithmetic &b \\
\hline
 -diag         & show diagnostic messages &b \\
\hline
 -exec        & interpret with no argument words &c \\
\hline
-explicit     & local variables must be explicitly declared &b \\
\hline
-format       & format output file (pretty-print) &b \\
\hline
-java         & generate Java source code for this program &b \\
\hline
 -keep         & keep any completed .java file (as xxx.java.keep) &c \\
\hline
-keepasjava   & keep any completed .java file (as xxx.java) &c \\
\hline
 -logo         & display logo (banner) after starting &b \\
\hline
-prompt       & prompt for new request after processing &c \\
\hline
-savelog      & save messages in NetRexxC.log &c \\
\hline
 -replace      & replace .java file even if it exists &b \\
\hline
 -sourcedir    & force output files to source directory &b \\
\hline
  -strictargs   & empty argument lists must be specified as () &b \\
\hline
  -strictassign & assignment must be cost-free &b \\
\hline
  -strictcase   & names must match in case &b \\
\hline
  -strictimport & all imports must be explicit &b \\
\hline
  -strictmethods & superclass methods are not compared to local methods for best match &b \\
\hline
  -strictprops  & even local properties must be qualified &b \\
\hline
  -strictsignal & signals list must be explicit &b \\
\hline
  -symbols      & include symbols table in generated .class files &b \\
\hline
-time         & display timings &c \\
\hline
 -trace[n]     & trace stream [1 or 2], or 0 for NOTRACE &b \\
\hline
 -utf8         & source file is in UTF8 encoding &b \\
\hline
 -verbose[n]   & verbosity of progress reports [0-5] &b \\
\hline
 -warnexit0    & exit with a zero returncode on warnings &c \\
\hline
\end{longtable}

\subsubsection{Options valid for the options statement and on the commandline}
These are the options that can be used on the \textbf{options} statement:
\begin{description}
\index{option, binary}
\index{flag, binary}
\index{binary option}
\item[-binary]
All classes in this program will be binary classes. In binary classes, literals are assigned binary (primitive) or native string types, rather than NetRexx types, and native binary operations are used to implement operators where appropriate, as described in “Binary values and operations”. In classes that are not binary, terms in expressions are converted to the NetRexx string type, Rexx, before use by operators.

\index{option,comments}
\index{flag,comments}
\index{comments option}
\item[-comments]
Comments from the NetRexx source program will be passed through to the Java output file (which may be saved with a .java.keep or .java extension by using the -keep and -keepasjava command options, respectively).

\index{option,compact}
\index{flag,compact}
\index{compact option}
\item[-compact]
Requests that warnings and error messages be displayed in compact form. This format is more easily parsed than the default format, and is intended for use by editing environments.
Each error message is presented as a single line, prefixed with the error token identification enclosed in square brackets. The error token identification comprises three words, with one blank separating the words. The words are: the source file specification, the line number of the error token, the column in which it starts, and its length. For example (all on one line):
\begin{verbatim}
  [D:\test\test.nrx 3 8 5] Error: The external name
  'class' is a Java reserved word, so would not be
  usable from Java programs
\end{verbatim}
Any blanks in the file specification are replaced by a null ('\textbackslash 0') character. Additional words could be added to the error token identification later.

\index{option,crossref}
\index{flag,crossref}
\index{crossref option}
\item[-crossref]
Requests that cross-reference listings of variables be prepared, by class.
\index{option,decimal}
\index{flag,decimal}
\index{decimal option}
\item[-decimal]
Decimal arithmetic may be used in the program. If nodecimal is specified, the language processor will report operations that use (or, like normal string comparison, might use) decimal arithmetic as an error. This option is intended for performance-critical programs where the overhead of inadvertent use of decimal arithmetic is unacceptable.
\index{option,diag}
\index{flag,diag}
\index{diag option}
\item[-diag]
Requests that diagnostic information (for experimental use only) be displayed. The diag option word may also have side-effects.
\index{option,explicit}
\index{flag,explicit}
\index{explicit option}
\item[-explicit]
Requires that all local variables must be explicitly declared (by assigning them a type but no value) before assigning any value to them. This option is intended to permit the enforcement of “house styles” (but note that the NetRexx compiler always checks for variables which are referenced before their first assignment, and warns of variables which are set but not used).
\index{option,format}
\index{flag,format}
\index{format option}
\item[-format]
Requests that the translator output file (Java source code) be formatted for improved readability. Note that if this option is in effect, line numbers from the input file will not be preserved (so run-time errors and exception trace-backs may show incorrect line numbers).
\index{option,java}
\index{flag,java}
\index{java option}
\item[-java]
Requests that Java source code be produced by the translator. If nojava is specified, no Java source code will be produced; this can be used to save a little time when checking of a program is required without any compilation or Java code resulting.
\index{option,logo}
\index{flag,logo}
\index{logo option}
\item[-logo]
Requests that the language processor display an introductory logotype sequence (name and version of the compiler or interpreter, etc.).
\index{option,sourcedir}
\index{flag,sourcedir}
\index{sourcedir option}
\item[-sourcedir]
Requests that all .class files be placed in the same directory as the source file from which they are compiled. Other output files are already placed in that directory. Note that using this option will prevent the -run command option from working unless the source directory is the current directory.
\index{option,strictargs}
\index{flag,strictargs}
\index{strictargs option}
\item[-strictargs]
Requires that method invocations always specify parentheses, even when no arguments are supplied. Also, if strictargs is in effect, method arguments are checked for usage – a warning is given if no reference to the argument is made in the method.
\index{option,strictassign}
\index{flag,strictassign}
\index{strictassign option}
\item[-strictassign]
Requires that only exact type matches be allowed in assignments (this is stronger than Java requirements). This also applies to the matching of arguments in method calls.
\index{option,strictcase}
\index{flag,strictcase}
\index{strictcase option}
\item[-strictcase]
Requires that local and external name comparisons for variables, properties, methods, classes, and special words match in case (that is, names must be identical to match).
\index{option,strictimport}
\index{flag,strictimport}
\index{strictimport option}
\item[-strictimport]
Requires that all imported packages and classes be imported explicitly using import instructions. That is, if in effect, there will be no automatic imports, except those related to the package instruction.
\index{option,strictmethods}
\index{flag,strictmethods}
\index{strictmethods option}
\item[-strictmethods]
Superclass methods are not compared to local methods for best match.
\index{option,strictprops}
\index{flag,strictprops}
\index{strictprops option}
\item[-strictprops]
Requires that all properties, including those local to the current class, be qualified in references. That is, if in effect, local properties cannot appear as simple names but must be qualified by this. (or equivalent) or the class name (for static properties).
\index{option,strictsignal}
\index{flag,strictsignal}
\index{strictsignal option}
\item[-strictsignal]
Requires that all checked exceptions signalled within a method but not caught by a catch clause be listed in the signals phrase of the method instruction.
\index{option,symbols}
\index{flag,symbols}
\index{symbols option}
\item[-symbols]
Symbol table information (names of local variables, etc.) will be included in any generated .class file. This option is provided to aid the production of classes that are easy to analyse with tools that can understand the symbol table information. The use of this option increases the size of .class files.
\index{option,trace, traceX}
\index{flag,trace, traceX}
\index{trace, traceX option}
\item[-trace, -traceX]
If given as \textbf{-trace}, \textbf{-trace1}, or \textbf{-trace2}, then trace instructions are accepted. The trace output is directed according to the option word: \textbf{-trace1} requests that trace output is written to the standard output stream, \textbf{-trace} or \textbf{-trace2} imply that the output should be written to the standard error stream (the default).
\index{option,utf8}
\index{flag,utf8}
\index{utf8 option}
\item[-utf8]
If given, clauses following the options instruction are expected to be encoded using UTF-8, so all Unicode characters may be used in the source of the program.
In UTF-8 encoding, Unicode characters less than '\textbackslash u0080' are represented using one byte (whose most-significant bit is 0), characters in the range '\textbackslash u0080' through '\textbackslash u07FF' are encoded as two bytes, in the sequence of bits:
\begin{verbatim}
  110xxxxx 10xxxxxx
\end{verbatim}
where the eleven digits shown as x are the least significant eleven bits of the character, and characters in the range '\textbackslash u0800' through '\textbackslash uFFFF' are encoded as three bytes, in the sequence of bits:
\begin{verbatim}
  1110xxxx 10xxxxxx 10xxxxxx
\end{verbatim}
where the sixteen digits shown as x are the sixteen bits of the character.
If noutf8 is given, following clauses are assumed to comprise only Unicode characters in the range '\textbackslash x00' through '\textbackslash xFF', with the more significant byte of the encoding of each character being 0.
Note: this option only has an effect as a compiler option, and applies to all programs being compiled. If present on an options instruction, it is checked and must match the compiler option (this allows processing with or without utf8 to be enforced).
\index{option,verbose, verboseX}
\index{flag,verbose, verboseX}
\index{verbose, verboseX option}
\item[-verbose, -verboseX]
Sets the “noisiness” of the language processor. The digit X may be any of the digits 0 through 5; if omitted, a value of 3 is used. The options \textbf{-noverbose} and \textbf{verbose0} both suppress all messages except errors and warnings
\end{description}

\subsubsection{Options valid on the commandline}
The translator also implements some additional option words, which
control compilation features.  These cannot be used on the
\textbf{options} instruction\footnote{Although at the moment, there will be no indication of this}, and are:
\begin{description}
\index{option,arg words}
\index{flag,arg words}
\index{arg words option}
\item[-keep]
The \textbf{-arg} \emph{words} option is used when interpreting programs, it indicates that after the \textbf{-arg} statement, commandline arguments for ther interpreted program follow.
\index{option,exec}
\index{flag,exec}
\index{exec option}
\item[-exec]
The \textbf{-exec} \emph{words} option is used when interpreting programs. With this option, no commandline arguments are possible.
\index{option,keep}
\index{flag,keep}
\index{keep option}
\item[-keep]
keep the intermediate \emph{.java} file for each program.  It is kept in
the same directory as the NetRexx source file as \emph{xxx.java.keep},
where \emph{xxx} is the source file name.  The file will also be kept
automatically if the \emph{javac} compilation fails for any reason.
\index{option,keepasjava}
\index{flag,keepasjava}
\index{keepasjava option}
\item[-keepasjava]
keep the intermediate \emph{.java} file for each program.  It is kept in
the same directory as the NetRexx source file as \emph{xxx.java},
where \emph{xxx} is the source file name.  Implies -replace. Note: use this option carefully in mixed-source projects where you might have .java source files around.
\item[-nocompile]
\index{option, nocompile}
\index{flag, nocompile}
\index{nocompile option}
do not compile (just translate).  Use this option when you want to use a
different Java compiler.  The \emph{.java} file for each program is kept
in the same directory as the NetRexx source file, as the
file \emph{xxx.java.keep} (where \emph{xxx} is the source file name).
\item[-noconsole]
\index{option, noconsole}
\index{flag, noconsole}
\index{noconsole option}
do not display compiler messages on the console (command display
screen).  This is usually used with the \emph{savelog} option.
\item[-savelog]
\index{option, savelog}
\index{flag, savelog}
\index{savelog option}
write compiler messages to the file \emph{NetRexxC.log}, in the current
directory.
This is often used with the \emph{noconsole} option.
\item[-time]
\index{option, time}
\index{flag, time}
\index{time option}
display translation, \emph{javac} or \emph{ecj} compile, and total times (for the sum
of all programs processed).
\item[-run]
\index{option, run}
\index{flag, run}
\index{run option}
run the resulting Java class as a stand-alone application, provided that
the compilation had no errors.
\index{option,warnexit0}
\index{flag,warnexit0}
\index{warnexit0 option}
\item[-warnexit0]
Exit the translator with returncode 0 even if warnings are issued. Useful with build tools that would otherwise exit a build.
\end{description}




Here are some examples:
\begin{verbatim}
java org.netrexx.process.NetRexxC hello -keep -strictargs
java org.netrexx.process.NetRexxC -keep hello wordclock
java org.netrexx.process.NetRexxC hello wordclock -nocompile
nrc hello
nrc hello.nrx
nrc -run hello
nrc -run Spectrum -keep
nrc hello -binary -verbose1
nrc hello -noconsole -savelog -format -keep
\end{verbatim}

Option words may be specified in lowercase, mixed case, or uppercase.
File specifications are platform-dependent and may be case sensitive,
though \nr{}C will always prefer an exact case match over a mismatch.

\textbf{Note:} The \emph{-run} option is implemented by a script (such
as \emph{nrc.bat} or \emph{\nr{}C.cmd}), not by the translator; some
scripts (such as the \emph{.bat} scripts) may require that
the \emph{-run} be the first word of the command arguments, and/or be in
lowercase.  They may also require that only the name of the file be
given if the \emph{-run} option is used.  Check the commentary at the
beginning of the script for details.

\section{Compiling multiple programs and using packages}
\index{compiling,multiple programs}

When you specify more than one program for \nr{}C to compile, they are
all compiled within the same class context: that is, they can see
the classes, properties, and methods of the other programs being
compiled, much as though they were all in one file.

This allows mutually interdependent programs and classes to be compiled
in a single operation.  For example, consider the following two programs
(assumed to be in your current directory, as the files \emph{X.nrx}
and \emph{Y.nrx}):
\begin{lstlisting}[label=dependencies,caption=Dependencies]
/* X.nrx */
class X
  why=Y null

/* Y.nrx */
class Y
  exe=X null
\end{lstlisting}
Each contains a reference to the other, so neither can be compiled in
isolation.  However, if you compile them together, using the command:
\begin{verbatim}
nrc X Y
\end{verbatim}
 the cross-references will be resolved correctly.

The total elapsed time will be significantly less, too, as the classes
on the CLASSPATH need to be located only once, and the class files used
by the \nr{}C compiler or the programs themselves will also only be
loaded (and JIT-compiled) once.

\index{projects, compiling}
\index{packages, compiling}
\index{compiling,packages}
This example works as you would expect for programs that are not in
packages.  There is a restriction, though, if the classes you are
compiling \emph{are} in packages (that is, they include a
\textbf{package} instruction).  \nr{}C uses either the \emph{javac}
compiler or the Eclipse batch compiler \emph{ecj} to generate the \emph{.class} files, and for mutually-dependent
files like these; both require the source files to be in the
Java \emph{CLASSPATH}, in the sub-directory described by the \textbf{package}
instruction.


So, for example, if your project is based on the tree:

\texttt{D:\textbackslash myproject}

 if the two programs above specified a package, thus:
\begin{lstlisting}[label=packagedep,caption=Package Dependencies]
/* X.nrx */
package foo.bar
class X
  why=Y null

/* Y.nrx */
package foo.bar
class Y
  exe=X null
\end{lstlisting}


\begin{enumerate}
\item
You should put these source files in the directory:
\emph{D:\textbackslash myproject\textbackslash foo\textbackslash bar}
\item
The directory \emph{D:\textbackslash myproject} should appear in your CLASSPATH
setting (if you don't do this, \emph{javac} will complain that it cannot
find one or other of the classes).
\item
You should then make the current directory be \emph{D:\textbackslash
myproject\textbackslash foo\textbackslash bar}
and then compile the programs using the command \emph{nrc X Y},
as above.
\end{enumerate}

With this procedure, you should end up with the \emph{.class} files in
the same directory as the \emph{.nrx} (source) files, and therefore also
on the CLASSPATH and immediately usable by other packages.  In general,
this arrangement is recommended whenever you are writing programs that
reside in packages.

\textbf{Notes:}
\begin{enumerate}
\item
When \emph{javac} is used to generate the \emph{.class} files, no
new \emph{.class} files will be created if any of the programs being
compiled together had errors - this avoids accidentally generating
mixtures of new and old \emph{.class} files that cannot work with each
other.
\item
If a class is abstract or is an adapter class then it should be placed
in the list before any classes that extend it (as otherwise any
automatically generated methods will not be visible to the subclasses).
\end{enumerate}

\chapter{Programmatic use of the \nr{}C translator}
\nr{}C can be used in a program, to compile \nr{} programs from files,
or to compile from strings in memory. 

\section{Compiling from memory strings}
Programs may also be compiled from memory strings by passing an array
of strings containing programs to the translator using these methods:

\begin{lstlisting}[label=frommemory,caption=From Memory]
method main(arg=Rexx, programarray=String[], log=PrintWriter null) static returns int
method main2(arg=String[], programarray=String[], log=PrintWriter null) static returns int
\end{lstlisting}

Any programs passed as strings must be named in the arg parameter before any programs contained in files are named.
For convenience when compiling a single program, the program can be
passed directly to the compiler as a String with this method:

\begin{lstlisting}[label=string,caption=With String argument]
method main(arg=Rexx, programstring=String, logfile=PrintWriter null) constant returns int
\end{lstlisting}

Here is an example of compiling a \nr{} program from a string in
memory:

\begin{lstlisting}[label=memexample,caption=Example of compiling from String]
import org.netrexx.process.NetRexxC
program = "say 'hello there via NetRexxC'"
NetRexxC.main("myprogram",program)
\end{lstlisting}

Other uses of the \nr{}A API are beyond the scope
of this Quick Start Guide and are documented in the \emph{Programming Guide}.